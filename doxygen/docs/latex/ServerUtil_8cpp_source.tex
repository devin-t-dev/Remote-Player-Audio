\hypertarget{ServerUtil_8cpp_source}{\section{\-Input/\-Socket/\-Server\-Util.cpp}
}

\begin{DoxyCode}
00001 
00002 
00003 
00004 \textcolor{preprocessor}{#ifndef SRVUTIL}
00005 \textcolor{preprocessor}{}\textcolor{preprocessor}{#define SVRUTIL}
00006 \textcolor{preprocessor}{}\textcolor{preprocessor}{#include <iostream>}
00007 \textcolor{preprocessor}{#include <stdio.h>}
00008 \textcolor{preprocessor}{#include <sys/types.h>}
00009 \textcolor{preprocessor}{#include <sys/socket.h>}
00010 \textcolor{preprocessor}{#include <netinet/in.h>}
00011 \textcolor{preprocessor}{#include <netdb.h>}
00012 \textcolor{preprocessor}{#include <string.h>}
00013 \textcolor{preprocessor}{#include <stdlib.h>}
00014 \textcolor{preprocessor}{#include <unistd.h>}
00015 \textcolor{preprocessor}{#include <errno.h>}
00016 
00017 \textcolor{keyword}{static} \textcolor{keywordtype}{int} sockfd, newsockfd, portno, clilen;
00018 \textcolor{keyword}{static} \textcolor{keyword}{struct }sockaddr\_in serv\_addr, cli\_addr;
00019 
00020 \textcolor{keyword}{static} \textcolor{keywordtype}{void} error( \textcolor{keywordtype}{char} *msg ) \{
00021   perror(  msg );
00022   exit(1);
00023 \}
00028 \textcolor{keyword}{static} \textcolor{keywordtype}{int} initServer(\textcolor{keywordtype}{int} port) \{
00029      portno = port;
00030      \textcolor{comment}{//char buffer[1024]; //To resize to right size every time as 'contenuto'}
00031      printf( \textcolor{stringliteral}{"\(\backslash\)n [I] Inizializzazione del Server (usando la porta #%d) \(\backslash\)n"}, 
      portno );
00032 
00033      sockfd = socket(AF\_INET, SOCK\_STREAM, 0);
00034      \textcolor{keywordflow}{if} (sockfd < 0)
00035          error( const\_cast<char *>(\textcolor{stringliteral}{"ERROR opening socket"}) );
00036      bzero((\textcolor{keywordtype}{char} *) &serv\_addr, \textcolor{keyword}{sizeof}(serv\_addr));
00037 
00038      serv\_addr.sin\_family = AF\_INET;
00039      serv\_addr.sin\_addr.s\_addr = INADDR\_ANY;
00040      serv\_addr.sin\_port = htons( portno );
00041      \textcolor{keywordflow}{if} (bind(sockfd, (\textcolor{keyword}{struct} sockaddr *) &serv\_addr, \textcolor{keyword}{sizeof}(serv\_addr)) < 0) \{
00042        \textcolor{comment}{/*error( const\_cast<char *>( "[E] ERROR on binding" ) );*/}
00043                  printf(\textcolor{stringliteral}{"\(\backslash\)n [E] Porta Attualmente in uso... Ritento"});
00044                  sleep(2);
00045                  initServer(portno);
00046                  \textcolor{keywordflow}{return} \textcolor{keyword}{false};
00047      \}
00048      listen(sockfd,5);
00049      clilen = \textcolor{keyword}{sizeof}(cli\_addr);
00050      \textcolor{keywordflow}{return} 0;
00051 \}
\hypertarget{ServerUtil_8cpp_source_l00055}{}\hyperlink{ServerUtil_8cpp_a7329f619144deb41baf72e83a7073701}{00055} \textcolor{keywordtype}{void} \hyperlink{ServerUtil_8cpp_a7329f619144deb41baf72e83a7073701}{mandaData}(\textcolor{keywordtype}{char} * data,\textcolor{keywordtype}{int} len)  \{
00056   \textcolor{keywordtype}{int} n;
00057   \textcolor{keywordtype}{char} buffer[len];
00058   sprintf( buffer, \textcolor{stringliteral}{"%s\(\backslash\)n"}, data );
00059   \textcolor{keywordflow}{if} ( (n = write(newsockfd, buffer, strlen(buffer) ) ) < 0 )
00060     error( const\_cast<char *>( \textcolor{stringliteral}{"ERROR writing to socket"}) );
00061   buffer[n] = \textcolor{charliteral}{'\(\backslash\)0'};
00062 \}
00063 
00065 \textcolor{comment}{/*}
00066 \textcolor{comment}{     Ottiene dal client un messaggio}
00067 \textcolor{comment}{ */}
\hypertarget{ServerUtil_8cpp_source_l00069}{}\hyperlink{ServerUtil_8cpp_af4ec6cff50fa0e44047464b9b8c912d4}{00069} \textcolor{comment}{/*static*/} \textcolor{keywordtype}{char} * \hyperlink{ServerUtil_8cpp_af4ec6cff50fa0e44047464b9b8c912d4}{ottieniData}() \{
00070   \textcolor{keywordtype}{char} buffer[1024];
00071   \textcolor{keywordtype}{int} n;
00072 
00073   \textcolor{keywordflow}{if} ( (n = read(newsockfd,buffer,31) ) < 0 )
00074     error( const\_cast<char *>( \textcolor{stringliteral}{"ERROR reading from socket"}) );
00075   buffer[n] = \textcolor{charliteral}{'\(\backslash\)0'};
00076   \textcolor{keywordflow}{return} buffer;
00077 \}
00078 \textcolor{comment}{/*}
00079 \textcolor{comment}{ * Rimane in attesa di un client}
00080 \textcolor{comment}{ */}
00081 \textcolor{keywordtype}{void} acceptClient() \{
00082 \textcolor{comment}{//--- wait on a connection ---}
00083     \textcolor{keywordflow}{if} ( ( newsockfd = accept( sockfd, (\textcolor{keyword}{struct} sockaddr *) &cli\_addr, (
      socklen\_t*) &clilen) ) < 0 )
00084         error( const\_cast<char *>(\textcolor{stringliteral}{"ERROR on accept"}) );
00085 \}
00086 \textcolor{comment}{/*}
00087 \textcolor{comment}{ * Funzione di chiusura della Socket}
00088 \textcolor{comment}{ */}
00089 \textcolor{keywordtype}{void} closeServer() \{
00090    close(newsockfd);
00091 \}
00092 \textcolor{comment}{/*}
00093 \textcolor{comment}{ * Funzione per chiudere pulire la ServerSocket}
00094 \textcolor{comment}{ */}
00095 \textcolor{keywordtype}{void} killServer() \{
00096     close(newsockfd);
00097     close(sockfd);
00098 \}
00099 \textcolor{comment}{/*}
00100 \textcolor{comment}{ * @name JSON Codifica}
00101 \textcolor{comment}{ * Questa funzione restituisce il messaggio correttamente formattato per
       l'invio del file JSON}
00102 \textcolor{comment}{ * @param message Messaggio da includere come valore. Questa \(\backslash\)b stringa verrà
       usata dal Client}
00103 \textcolor{comment}{ * @param type \(\backslash\)b Stringa che indica il tipo del messaggio. Usato dal client
       prevalentemente per motivi di \(\backslash\)b debug}
00104 \textcolor{comment}{ */}
00105 std::string JSONEncode(std::string message, std::string type) \{
00106     \textcolor{comment}{/* Example of a JSON file}
00107 \textcolor{comment}{     * \{}
00108 \textcolor{comment}{    "type": "menu",}
00109 \textcolor{comment}{    "value": "File",}
00110 \textcolor{comment}{    "items": [}
00111 \textcolor{comment}{        \{"value": "New", "action": "CreateNewDoc"\},}
00112 \textcolor{comment}{        \{"value": "Open", "action": "OpenDoc"\},}
00113 \textcolor{comment}{        \{"value": "Close", "action": "CloseDoc"\}}
00114 \textcolor{comment}{    ]}
00115 \textcolor{comment}{\}}
00116 \textcolor{comment}{     */}
00117     std::string partialEncode;
00118     partialEncode+=\textcolor{stringliteral}{"\{"};
00119     partialEncode+=\textcolor{stringliteral}{"\(\backslash\)"app\(\backslash\)": 4,"};
00120     partialEncode+=\textcolor{stringliteral}{"\(\backslash\)"type\(\backslash\)": \(\backslash\)""}+type+\textcolor{stringliteral}{"\(\backslash\)","};
00121     partialEncode+=\textcolor{stringliteral}{"\(\backslash\)"value\(\backslash\)": "}+message;
00122     partialEncode+=\textcolor{stringliteral}{"\}"};
00123     \textcolor{keywordflow}{return} partialEncode;
00124 \}
00125 \textcolor{preprocessor}{#endif}
\end{DoxyCode}
